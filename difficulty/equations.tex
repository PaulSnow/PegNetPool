\documentclass[12pt]{article}
\usepackage{mathtools}

\DeclarePairedDelimiter\ceil{\lceil}{\rceil}
\DeclarePairedDelimiter\floor{\lfloor}{\rfloor}

\setlength{\oddsidemargin}{0in}
\setlength{\evensidemargin}{\oddsidemargin}
\setlength{\textwidth}{6.5in}
\setlength{\textheight}{9in}
\setlength{\topmargin}{-0.25in}
\newcommand{\bb}{$\Box$}
\newcommand{\sline} {$\mbox{\underline{\hspace{4cm}}}$}
\newcommand{\lline} {$\mbox{\underline{\hspace{4cm}}}$}
\newcommand{\llline} {$\mbox{\underline{\hspace{8cm}}}$}
\newcommand{\textmathbf}[1]{\textbf{\boldmath#1}}
\usepackage{amsmath}
\usepackage{algorithm}
\usepackage[noend]{algpseudocode}
\usepackage{enumitem}% http://ctan.org/pkg/enumitem




\begin{document}
\begin{center}
{\bf
Pegnet Proof of Work \\
Target \& Difficulty Equations\\
}

\end{center}

\section{Definitions}

\begin{description}[font=\sffamily\bfseries, leftmargin=1cm, style=nextline]
    \item[Hash-rate]
    An amount of lxr hashing operations per second
    \item[Target]
    8 byte hexadecimal value that comes from the top 8 bytes of the lxr(oprhash+nonce). The higher the target, the more difficult that target is.
    \item[Difficulty]
    Difficulty is a floating point number that can describe the amount of work needed to generate a given target. The difficulty is proportional to the hashrate, meaning a target with difficulty 2 requires twice as many hashes (on average) as a target with difficulty 1. All difficulty numbers are related to a base difficulty 1, which is arbitrarily chosen.
    \item[pDiff]
    pDiff is the base difficulty for determining the difficulty of work submitted to a mining pool. pDiff is currently set at \texttt{0xfffe000000000000}, which is about 6.5K h/s for 5s of work. All difficulty reported by the pool is in relation to pDiff. Meaning a difficulty of 100 is 100x the work of pDiff.
\end{description}

\pagebreak
\section{Equations}

\subsection{Work}

The amount of work for a given target should be denoted in \textbf{difficulty}, but difficulty is not necessarily universal if a different base is chosen. Miners are more familiar with hashrate.


   \subsubsection{Total-Hashes from Target}
    To get the total number of hashes that on average will obtain the given target: \\
    \large{\[
    TH = \frac{2^{64}}{2^{64}-target}
    \]}

    \subsubsection{Difficulty from Target}
    To get the difficulty of a given target:
    \large{\[
    diff = \frac{\neg pDiff}{\neg target}
    \]}

    \subsubsection{Estimate target from number of hashes}
    Given a total number of hashes, we can estimate the best target.\\ Where $TH = totalhashes$
     \large{\[
    target=\frac{2^{64}*(TH-1)}{TH}
    \]}


    \subsubsection{Target from Difficulty}
    To get the target of a given difficulty:
    \large{\[
    \neg target = \frac{\neg pDiff}{\neg diff}
    \]}



\end{document}